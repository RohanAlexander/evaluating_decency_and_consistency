% Options for packages loaded elsewhere
\PassOptionsToPackage{unicode}{hyperref}
\PassOptionsToPackage{hyphens}{url}
\PassOptionsToPackage{dvipsnames,svgnames,x11names}{xcolor}
%
\documentclass[
  letterpaper,
  DIV=11,
  numbers=noendperiod]{scrartcl}

\usepackage{amsmath,amssymb}
\usepackage{iftex}
\ifPDFTeX
  \usepackage[T1]{fontenc}
  \usepackage[utf8]{inputenc}
  \usepackage{textcomp} % provide euro and other symbols
\else % if luatex or xetex
  \usepackage{unicode-math}
  \defaultfontfeatures{Scale=MatchLowercase}
  \defaultfontfeatures[\rmfamily]{Ligatures=TeX,Scale=1}
\fi
\usepackage{lmodern}
\ifPDFTeX\else  
    % xetex/luatex font selection
\fi
% Use upquote if available, for straight quotes in verbatim environments
\IfFileExists{upquote.sty}{\usepackage{upquote}}{}
\IfFileExists{microtype.sty}{% use microtype if available
  \usepackage[]{microtype}
  \UseMicrotypeSet[protrusion]{basicmath} % disable protrusion for tt fonts
}{}
\makeatletter
\@ifundefined{KOMAClassName}{% if non-KOMA class
  \IfFileExists{parskip.sty}{%
    \usepackage{parskip}
  }{% else
    \setlength{\parindent}{0pt}
    \setlength{\parskip}{6pt plus 2pt minus 1pt}}
}{% if KOMA class
  \KOMAoptions{parskip=half}}
\makeatother
\usepackage{xcolor}
\setlength{\emergencystretch}{3em} % prevent overfull lines
\setcounter{secnumdepth}{5}
% Make \paragraph and \subparagraph free-standing
\ifx\paragraph\undefined\else
  \let\oldparagraph\paragraph
  \renewcommand{\paragraph}[1]{\oldparagraph{#1}\mbox{}}
\fi
\ifx\subparagraph\undefined\else
  \let\oldsubparagraph\subparagraph
  \renewcommand{\subparagraph}[1]{\oldsubparagraph{#1}\mbox{}}
\fi


\providecommand{\tightlist}{%
  \setlength{\itemsep}{0pt}\setlength{\parskip}{0pt}}\usepackage{longtable,booktabs,array}
\usepackage{calc} % for calculating minipage widths
% Correct order of tables after \paragraph or \subparagraph
\usepackage{etoolbox}
\makeatletter
\patchcmd\longtable{\par}{\if@noskipsec\mbox{}\fi\par}{}{}
\makeatother
% Allow footnotes in longtable head/foot
\IfFileExists{footnotehyper.sty}{\usepackage{footnotehyper}}{\usepackage{footnote}}
\makesavenoteenv{longtable}
\usepackage{graphicx}
\makeatletter
\def\maxwidth{\ifdim\Gin@nat@width>\linewidth\linewidth\else\Gin@nat@width\fi}
\def\maxheight{\ifdim\Gin@nat@height>\textheight\textheight\else\Gin@nat@height\fi}
\makeatother
% Scale images if necessary, so that they will not overflow the page
% margins by default, and it is still possible to overwrite the defaults
% using explicit options in \includegraphics[width, height, ...]{}
\setkeys{Gin}{width=\maxwidth,height=\maxheight,keepaspectratio}
% Set default figure placement to htbp
\makeatletter
\def\fps@figure{htbp}
\makeatother
\newlength{\cslhangindent}
\setlength{\cslhangindent}{1.5em}
\newlength{\csllabelwidth}
\setlength{\csllabelwidth}{3em}
\newlength{\cslentryspacingunit} % times entry-spacing
\setlength{\cslentryspacingunit}{\parskip}
\newenvironment{CSLReferences}[2] % #1 hanging-ident, #2 entry spacing
 {% don't indent paragraphs
  \setlength{\parindent}{0pt}
  % turn on hanging indent if param 1 is 1
  \ifodd #1
  \let\oldpar\par
  \def\par{\hangindent=\cslhangindent\oldpar}
  \fi
  % set entry spacing
  \setlength{\parskip}{#2\cslentryspacingunit}
 }%
 {}
\usepackage{calc}
\newcommand{\CSLBlock}[1]{#1\hfill\break}
\newcommand{\CSLLeftMargin}[1]{\parbox[t]{\csllabelwidth}{#1}}
\newcommand{\CSLRightInline}[1]{\parbox[t]{\linewidth - \csllabelwidth}{#1}\break}
\newcommand{\CSLIndent}[1]{\hspace{\cslhangindent}#1}

\usepackage{booktabs}
\usepackage{longtable}
\usepackage{array}
\usepackage{multirow}
\usepackage{wrapfig}
\usepackage{float}
\usepackage{colortbl}
\usepackage{pdflscape}
\usepackage{tabu}
\usepackage{threeparttable}
\usepackage{threeparttablex}
\usepackage[normalem]{ulem}
\usepackage{makecell}
\usepackage{xcolor}
\usepackage{siunitx}

  \newcolumntype{d}{S[
    input-open-uncertainty=,
    input-close-uncertainty=,
    parse-numbers = false,
    table-align-text-pre=false,
    table-align-text-post=false
  ]}
  
\KOMAoption{captions}{tableheading}
\usepackage{listings}
\lstset{basicstyle=\footnotesize\ttfamily,breaklines=true}
\makeatletter
\makeatother
\makeatletter
\makeatother
\makeatletter
\@ifpackageloaded{caption}{}{\usepackage{caption}}
\AtBeginDocument{%
\ifdefined\contentsname
  \renewcommand*\contentsname{Table of contents}
\else
  \newcommand\contentsname{Table of contents}
\fi
\ifdefined\listfigurename
  \renewcommand*\listfigurename{List of Figures}
\else
  \newcommand\listfigurename{List of Figures}
\fi
\ifdefined\listtablename
  \renewcommand*\listtablename{List of Tables}
\else
  \newcommand\listtablename{List of Tables}
\fi
\ifdefined\figurename
  \renewcommand*\figurename{Figure}
\else
  \newcommand\figurename{Figure}
\fi
\ifdefined\tablename
  \renewcommand*\tablename{Table}
\else
  \newcommand\tablename{Table}
\fi
}
\@ifpackageloaded{float}{}{\usepackage{float}}
\floatstyle{ruled}
\@ifundefined{c@chapter}{\newfloat{codelisting}{h}{lop}}{\newfloat{codelisting}{h}{lop}[chapter]}
\floatname{codelisting}{Listing}
\newcommand*\listoflistings{\listof{codelisting}{List of Listings}}
\makeatother
\makeatletter
\@ifpackageloaded{caption}{}{\usepackage{caption}}
\@ifpackageloaded{subcaption}{}{\usepackage{subcaption}}
\makeatother
\makeatletter
\@ifpackageloaded{tcolorbox}{}{\usepackage[skins,breakable]{tcolorbox}}
\makeatother
\makeatletter
\@ifundefined{shadecolor}{\definecolor{shadecolor}{rgb}{.97, .97, .97}}
\makeatother
\makeatletter
\makeatother
\makeatletter
\makeatother
\ifLuaTeX
  \usepackage{selnolig}  % disable illegal ligatures
\fi
\IfFileExists{bookmark.sty}{\usepackage{bookmark}}{\usepackage{hyperref}}
\IfFileExists{xurl.sty}{\usepackage{xurl}}{} % add URL line breaks if available
\urlstyle{same} % disable monospaced font for URLs
\hypersetup{
  pdftitle={Evaluating the Decency and Consistency of Data Validation Tests Generated by LLMs},
  pdfauthor={Rohan Alexander; Lindsay Katz; Callandra Moore; Zane Schwartz},
  colorlinks=true,
  linkcolor={blue},
  filecolor={Maroon},
  citecolor={Blue},
  urlcolor={Blue},
  pdfcreator={LaTeX via pandoc}}

\title{Evaluating the Decency and Consistency of Data Validation Tests
Generated by LLMs\thanks{We thank Monica Alexander and Chris Maddison
for helpful suggestions. Code and data:
https://github.com/RohanAlexander/evaluating\_decency\_and\_consistency.
Contributions: RA had the initial idea. RA and CM developed the
experimental set-up. ZS established the political donations datasets. LK
developed the initial suite of dataset validation tests that we compare
the LLMs against. RA wrote the code to interact with the LLMs, evaluated
the LLM outputs, and did the modelling. All authors contributed to
writing the initial draft as well as improving and finalizing the
paper.}}
\author{Rohan Alexander\footnote{Information and Statistical Sciences,
  University of Toronto. rohan.alexander@utoronto.ca.} \and Lindsay
Katz\footnote{The Investigative Journalism Foundation} \and Callandra
Moore\footnote{The Investigative Journalism Foundation} \and Zane
Schwartz\footnote{The Investigative Journalism Foundation}}
\date{October 2, 2023}

\begin{document}
\maketitle
\begin{abstract}
We investigated the potential of large language models (LLMs) in
developing dataset validation tests. We carried out 96 experiments each
for both GPT-3.5 and GPT-4, examining different prompt scenarios,
learning modes, temperature settings, and roles. The prompt scenarios
were: 1) Asking for expectations, 2) Asking for expectations with a
given context, 3) Asking for expectations after requesting a simulation,
and 4) Asking for expectations with a provided data sample. For learning
modes, we tested: 1) zero-shot, 2) one-shot, and 3) few-shot learning.
We also tested four temperature settings: 0, 0.4, 0.6, and 1.
Furthermore, two distinct roles were considered: 1) ``helpful
assistant'', 2) ``expert data scientist''. To gauge consistency, every
setup was tested five times. The LLM-generated responses were
benchmarked against a gold standard suite, created by an experienced
data scientist knowledgeable about the data in question. We find there
are considerable returns to the use of few-shot learning, and that the
more explicit the data setting can be the better. The best LLM
configurations complement, rather than substitute, the gold standard
results. This study underscores the value LLMs can bring to the data
cleaning and preparation stages of the data science workflow.
\end{abstract}
\ifdefined\Shaded\renewenvironment{Shaded}{\begin{tcolorbox}[boxrule=0pt, breakable, interior hidden, borderline west={3pt}{0pt}{shadecolor}, enhanced, sharp corners, frame hidden]}{\end{tcolorbox}}\fi

\hypertarget{introduction}{%
\section{Introduction}\label{introduction}}

The Investigative Journalism Foundation (IJF) created and maintains a
dataset related to political donations in Canada. As of September 2023,
the dataset comprises 9,204,112 observations and 15 variables. Every day
new observations are added, based on newly released donations records
made available by provincial and federal elections agencies. This
release cycle is variable, with periods of inactivity followed by bursts
of multiple releases. Katz and Moore (2023) manually construct an
extensive suite of automated tests for this dataset. These impose
certain minimum standards on the dataset, including: that constituent
aspects add to match any total, class is appropriate, and null values
are where expected. This suite allows researchers to use the dataset
with confidence and ensures that new additions are fit for purpose.

We revisit this suite of tests to determine whether we can use large
language models (LLMs) to mimic this suite of validation tests. We
consider a variety of prompts, roles, learning modes, and temperature
settings, resulting in 96 total experiments. In particular, we consider
four prompt variations: ask for expectations; ask for expectations given
described context; ask for expectations having first asked for a
simulation; and ask for expectations given a sample of data. We also
consider zero-, one-, and few-shot learning; four temperature values: 0,
0.4, 0.6, and 1; and two roles: helpful assistant and expert data
scientist. For every combination we obtain five responses from the LLM.
We run all experiments separately for both GPT-3.5 and GPT-4.

A human coder judges the responses produced by the LLMs, rating their
decency (1-5, where 1 is the worst and 5 is the best) and their
consistency (1-5, where 1 means they are very different and 5 means they
are essentially identical). We then build an ordinal regression model
with \texttt{rstanarm} to explore the relationships that decency and
consistency have with prompts, roles, learning modes, and temperature
settings.

We find there are considerable returns to one- and few-shot learning. We
also find that including detailed descriptions of the expected dataset
can help improve the quality of the tests that are produced, even more
than including an example of the dataset. Temperature is associated with
consistency with not with decency. Surprisingly, there was not much of a
difference found between GPT-3.5 and GPT-4.

Our results demonstrate one use for LLMs in a data analysis workflow,
outside of just analysis. In particular, data scientists are often
concerned that results may be an artifact of some errors in the data
cleaning and preparation pipeline. Data validation tests help assuage
these concerns but they can be time consuming to produce. This paper
shows it is possible to use LLMs to establish an initial suite of tests,
which may encourage their use.

The remainder of this paper is structured as follows:
Section~\ref{sec-background} provides a brief overview of the underlying
dataset about which we are writing tests, as well as LLMs.
Section~\ref{sec-data} details the human coded LLM responses, which is
the analysis dataset used in this paper. Section~\ref{sec-model}
specifies the analysis model used and Section~\ref{sec-results} details
the estimates. Finally Section~\ref{sec-discussion} discusses some of
the implications of this study and details a few weaknesses.

\hypertarget{sec-background}{%
\section{Background}\label{sec-background}}

\hypertarget{the-political-donations-dataset}{%
\subsection{The political donations
dataset}\label{the-political-donations-dataset}}

The Investigative Journalism Foundation (IJF) is a nonprofit news media
outlet that is centered around public interest journalism. Their mandate
is to help rebuild trust in Canadian democracy and hold the powerful
accountable through data-driven investigative reporting. A core
component of the IJF's work is building and actively maintaining eight
publicly available databases with information on political donations,
registered charities, and political lobbying in Canada. While the
information that these databases are built upon are ostensibly public,
they are not maintained or available in a way that is widely accessible
or conducive to analysis. Further, the format and accessibility of these
data vary over time and across jurisdictions, making it difficult to
look at temporal or regional differences in the data. In turn, the IJFs
collation of these databases and high-quality journalism informed by
these data serves as a crucial contribution to rebuilding trust and
transparency in Canadian democracy.

One of the IJFs eight databases, and the focus of this work, is the
political donations database. Canadian legislation requires political
parties and candidates to disclose records of financial contributions
they receive. These records are maintained by elections agencies across
provinces and territories, and at the federal level. The frequency and
scope of these disclosures vary across jurisdictions. For instance, in
British Columbia, parties, candidates, constituency associations, and
leadership and nomination contestants can receive political donations,
while in Newfoundland and Labrador, donation recipients are limited to
only parties and candidates (The IJF 2023). The IJF's political
donations database is a compilation of these political finance records
across all Canadian jurisdictions, with data spanning from 1993 to the
present day. The database contains 15 variables including the donor's
name, the political party and entity to which the donation was made, the
amount donated, as well as the region and year of the donation.

While the IJF's database is available in a clean, user-friendly format,
the original records upon which it was created were not all accessible
in this way. The format of donation records varies across jurisdiction
and time. While some are available in readily downloadable spreadsheets,
others are available as PDF or HTML files --- the former necessitating
the use of optical character recognition (OCR) (The IJF 2023). To
prepare their database for publication, the IJF team performed
significant manual cleaning. The majority of this work resulted from the
conversion of PDF donation records to rectangular CSV format using OCR.
This is prone to scanning-related errors, such as the number 0 being
scanned as the letter O. The IJF manually corrected these errors
wherever they were identified by comparing the machine-legible OCR
output to the original PDF donations record (The IJF 2023).

Additional cleaning was done for the purpose of data legibility. For
instance, the IJF standardized donation dates to match the YYYY-MM-DD
format, and they standardized donor names which were in the format
``Surname, First name'' to be in the form of ``First name Surname'' (The
IJF 2023). Party names and donor types were also standardized for
consistency, and donation records with an abbreviated party name were
supplied with the complete name for that party. Finally, in rows where
the donor type was null but only individuals were legally allowed to
make donations in that jurisdiction and year, the IJF changed that null
entry to be ``Individual'' (The IJF 2023).

Despite the cleaning performed by the IJF team, the magnitude of these
data coupled with their self-reported nature makes them prone to both
human and computational error arising from the parsing process. With
over 9.2 million rows in this database, it would be a massive
undertaking for the IJF to manually identify all errors and
inconsistencies present in their data --- whether that is as minor as an
incorrectly formatted date, or as major as a reported donation amount
thousands of dollars above the legal limit. To address this challenge,
the IJF uses computational tools to assess data quality (Katz and Moore
2023). The team has developed a suite of comprehensive data tests for
all eight databases using Python's \texttt{Great\ Expectations} library.
This contains a number of pre-defined functions to test that particular
characteristics expected of a dataset hold true in the data at large.

As described by Katz and Moore (2023), the process of developing this
test suite was iterative in nature and necessitated significant domain
knowledge to develop accurate and valuable tests. For the political
donations database, Katz and Moore (2023) developed a comprehensive
suite of tests pertaining to missingness, formatting, and value
expectations in the data. For instance, they set the expectation that
for donations made in British Columbia, Ontario, or federally, the
\texttt{donation\_date} entry should not be missing, because those are
the only three jurisdictions which collect that variable (Katz and Moore
2023). They also test that donations data from 2022 align with the 2022
legal donation limits for each jurisdiction, and that, where applicable,
the sum of reported monetary and non-monetary contribution amounts add
up to the total reported donation amount. Katz and Moore (2023) provide
complete details on the development and implementation of these tests.

\hypertarget{large-language-models}{%
\subsection{Large language models}\label{large-language-models}}

Large language models (LLMs) are customized and refined through
in-context learning using prompting (Ouyang et al. 2023). A prompt is a
set of natural language instructions which define the parameters and
context for the desired output and may include one or more input-output
examples. By using this approach, LLMs can apply their existing
knowledge gained from training on various datasets to adapt to new
contexts specified by the prompt. The outcomes generated by LLMs are
sensitive to the specific phrasing and structure of these natural
language instructions. Consequently, there is ongoing work to establish
effective prompt patterns (White et al. 2023).

\hypertarget{sec-data}{%
\section{Data}\label{sec-data}}

We are interested in the extent to which the LLMs can develop a suite of
data validation tests that is similar to a suite developed by an
experienced expert data scientist who is familiar with the dataset. To
test this, we establish and run a series of experiments where we
consider different specifications and then compare the LLM output. In
particular, the variables that we consider are:

\begin{itemize}
\tightlist
\item
  Four prompts:

  \begin{lstlisting}
      - The Investigative Journalism Foundation (IJF) created and maintains a CSV dataset related to political donations in Canada. Each observation in the dataset is a donation, and the dataset has the following variables: "index", "amount", "donor_location", "donation_date", "donor_full_name", "donor_type", "political_entity", "political_party", "recipient", "region", "donation_year", "amount_monetary", "amount_non_monetary", "electoral_event", "electoral_district", "added".\n\rPlease write a series of expectations using the Python package great_expectations for this dataset.
      - The Investigative Journalism Foundation (IJF) created and maintains a CSV dataset related to political donations in Canada. Each observation in the dataset is a donation, and the dataset has the following variables: "index", "amount", "donor_location", "donation_date", "donor_full_name", "donor_type", "political_entity", "political_party", "recipient", "region", "donation_year", "amount_monetary", "amount_non_monetary", "electoral_event", "electoral_district", "added". 
          - "amount" is a monetary value that cannot be less than $0. An example observation is "195.46". It is, possible, but unlikely to be more than $1,000.00. It cannot be NA. It should be a numeric. The maximum donation "amount" depends on the value of "region" and "year". For "Federal" is 1675, for "Quebec" is 100 since 2013 and 500 for earlier years, for "British Columbia" is 1309.09, for "Ontario" is 3325, and for "Alberta" is 4300. There is no limit for "Saskatchewan".
          - "amount" should be equal to the sum of "amount_monetary" and "amount_non_monetary".
          - "region" can be one of the following values: "Federal", "Quebec", "British Columbia", "Ontario", "Saskatchewan", "Alberta". It cannot be NA. It should be a factor variable.
          - "donor_full_name" is a string. It cannot be NA. It is usually a first and last name, but might also include a middle initial. It should be in title case.
          - "donation_date" should be a date in the following format: YYYY-MM-DD. It could be NA. The earliest donation is from 2010-01-01. The latest donation is from 2023-09-01.
          - "donation_year" should match the year of "donation_date" if "donation_date" is not NA, but it is possible that "donation_year" exists even if "donation_date" does not. The earliest year is 2010 and the latest year is 2023. This variable is an integer.
          - "political_party" cannot be NA. It should be a factor that is equal to one of: "New Democratic Party", "Liberal Party of Canada", "Conservative Party of Canada".

          Please write a series of expectations using the Python package great_expectations for this dataset.
      - The Investigative Journalism Foundation (IJF) created and maintains a CSV dataset related to political donations in Canada. Each observation in the dataset is a donation, and the dataset has the following variables: "index", "amount", "donor_location", "donation_date", "donor_full_name", "donor_type", "political_entity", "political_party", "recipient", "region", "donation_year", "amount_monetary", "amount_non_monetary", "electoral_event", "electoral_district", "added". 
          - "amount" is a monetary value that cannot be less than $0. An example observation is "195.46". It is, possible, but unlikely to be more than $1,000.00. It cannot be NA. It should be a numeric. The maximum donation "amount" depends on the value of "region" and "year". For "Federal" is 1675, for "Quebec" is 100 since 2013 and 500 for earlier years, for "British Columbia" is 1309.09, for "Ontario" is 3325, and for "Alberta" is 4300. There is no limit for "Saskatchewan".
          - "amount" should be equal to the sum of "amount_monetary" and "amount_non_monetary".
          - "region" can be one of the following values: "Federal", "Quebec", "British Columbia", "Ontario", "Saskatchewan", "Alberta". It cannot be NA. It should be a factor variable.
          - "donor_full_name" is a string. It cannot be NA. It is usually a first and last name, but might also include a middle initial. It should be in title case.
          - "donation_date" should be a date in the following format: YYYY-MM-DD. It could be NA. The earliest donation is from 2010-01-01. The latest donation is from 2023-09-01.
          - "donation_year" should match the year of "donation_date" if "donation_date" is not NA, but it is possible that "donation_year" exists even if "donation_date" does not. The earliest year is 2010 and the latest year is 2023. This variable is an integer.
          - "political_party" cannot be NA. It should be a factor that is equal to one of: "New Democratic Party", "Liberal Party of Canada", "Conservative Party of Canada".

          Please simulate an example dataset of 1000 observations. Based on that simulation please write a series of expectations using the Python package great_expectations for this dataset.
      - The Investigative Journalism Foundation (IJF) created and maintains a CSV dataset related to political donations in Canada. Each observation in the dataset is a donation, and the dataset has the following variables: "index", "amount", "donor_location", "donation_date", "donor_full_name", "donor_type", "political_entity", "political_party", "recipient", "region", "donation_year", "amount_monetary", "amount_non_monetary", "electoral_event", "electoral_district", "added". 
          An example of a dataset is: 
          index,amount,donor_location,donation_date,donor_full_name,donor_type,political_entity,political_party,recipient,region,donation_year,amount_monetary,amount_non_monetary,electoral_event,electoral_district,added
          5279105,$20.00,"Granton, N0M1V0",2014-08-15,Shelley Reynolds,Individual,Party,New Democratic Party,New Democratic Party,Federal,2014,20.0,0.0,Annual,Nan,2022-11-23 01:00:31.771769+00:00
          2187800,$200.00,,,Robert Toupin,Individual,Party,Coalition Avenir  Legault,,Quebec,2018,,,,,2023-03-17 18:02:29.706250+00:00
          3165665,$50.00,,,Genevive Dussault,Individual,Party,Qubec Solidaire  (Avant Fusion),,Quebec,2017,,,,,2023-03-19 18:02:24.746621+00:00
          8803473,$250.00,"Nan, Nan",,Roger Anderson,Individual,Party,Reform Party Of Canada,Reform Party Of Canada,Federal,1994,0.0,0.0,Annual,Nan,2022-11-22 02:25:34.868056+00:00
          2000776,"$1,425.00","Calgary, T3H5K2",2018-10-30,Melinda Parker,Individual,Registered associations,Liberal Party Of Canada,Calgary Centre Federal Liberal Association,Federal,2018,1425.0,0.0,Annual,Calgary Centre,2022-11-23 01:00:31.771769+00:00
          9321613,$75.00,,2022-06-17,Jeffrey Andrus,Individual,Party,Bc Ndp,Bc Ndp,British Columbia,2022,,,,,2022-12-21 02:20:49.009276+00:00
          2426288,$50.00,"Stony Plain, T7Z1L5",2018-07-24,Phillip L Poulin,Individual,Party,Conservative Party Of Canada,Conservative Party Of Canada,Federal,2018,50.0,0.0,Annual,Nan,2022-11-23 01:00:31.771769+00:00
          4428629,$100.00,"Calgary, T2Y4K1",2015-07-30,Barry Hollowell,Individual,Party,New Democratic Party,New Democratic Party,Federal,2015,100.0,0.0,Annual,Nan,2022-11-23 01:00:31.771769+00:00
          1010544,$20.00,"Langley, V1M1P2",2020-05-31,Carole Sundin,Individual,Party,New Democratic Party,New Democratic Party,Federal,2020,20.0,0.0,Annual,Nan,2022-11-23 01:00:31.771769+00:00
          4254927,$500.00,"Welshpool, E5E1Z1",2015-10-10,Melville E Young,Individual,Party,Conservative Party Of Canada,Conservative Party Of Canada,Federal,2015,500.0,0.0,Annual,Nan,2022-11-23 01:00:31.771769+00:00
          8001740,$90.00,"Deleau, R0M0L0",2004-11-15,Clarke Robson,Individual,Party,New Democratic Party,New Democratic Party,Federal,2004,90.0,0.0,Annual,Nan,2022-11-23 01:00:31.771769+00:00

          Based on this sample please write a series of expectations using the Python package great_expectations for this dataset.
  \end{lstlisting}
\item
  Three learning modes: zero-, one-, and few-shot learning:

  \begin{lstlisting}
      - The following text in quotes is an example of an expectation for this dataset:
          """
          # Check that there is nothing null in any column of donations details
          donations_mv.expect_column_values_to_not_be_null(column='donor_full_name')
          """
      - The following text in quotes is an example of three expectations for this dataset:
          """
          # Check that there is nothing null in any column of donations details
          donations_mv.expect_column_values_to_not_be_null(column='donor_full_name')
          # Check that the federal donation does not exceed the maximum
          donations_mv.expect_column_values_to_be_between(
              column = 'amount',
              max_value = 1675,
              row_condition = 'region=="Federal" & donor_full_name.str.contains("Contributions Of")==False & donor_full_name.str.contains("Estate Of")==False & donor_full_name.str.contains("Total Anonymous Contributions")==False & donation_year == 2022 & political_entity.str.contains("Leadership")==False',
              condition_parser = 'pandas'
          )
          # Check that the date matches an appropriate regex format
          donations_mv.expect_column_values_to_match_regex(column = 'donation_date',
                                                      regex = '\\d{4}-\\d{2}-\\d{2}',
                                                      row_condition = "donation_date.isna()==False",
                                                      condition_parser = 'pandas')
          """
  \end{lstlisting}
\item
  Four temperature values: 0, 0.4, 0.6, and 1.
\item
  Two roles:

  \begin{lstlisting}
      - You are a helpful assistant.
      - You are a highly-trained, experienced, data scientist who is an expert at writing readable, correct, Python code.
  \end{lstlisting}
\end{itemize}

This combination of variables and options results in 96 different prompt
situations. We run these through both GPT-3.5 and GPT-4, using the API.
For every combination we ask for five responses to understand variation.

This results in a dataset of responses. The option that gave rise to
each response was blinded and the order randomized, and then the
responses were ranked by one experienced human coder on two metrics. The
first, ``consistency'', is a ranking 1-5 of how different each of the
five responses was for that particular combination of variables. 1 means
that responses 1-5 were wildly different. 5 means that responses 1-5 are
entirely or essentially the same. The human coder then ranked the
``decency'' of the first response for each combination of variables.
This is a measure of how effective the LLM validation tests were
compared with the code written by the experienced data scientist who
wrote the original suite of tests. The LLM responses are not expected to
have the full context of the code, so we do not expect an exact match,
but it should actually write code, import relevant libraries, add
comments, deal with class, and write a variety of relevant expectations.
1 means that the code is unusable, 2 means that it is not unusable but
would need a lot of work and would be disappointing from a human, 3
means that it is fine but would need some fixing, 4 means it is broadly
equivalent to what the gold standard suite contains, and 5 means it is
in no worse and is better in some way than the gold standard validation
suite.

Figure~\ref{fig-version} examines how decency and consistency differ
based on whether GPT-3.5 or GPT-4 is used. Unexpectedly, GPT-4 has fewer
responses rated 5/5 for decency, compared with GPT-3.5
(Figure~\ref{fig-version-1}). GPT-3.5 has fewer responses rated 2/5, but
overall the mean decency response for GPT-3.5 is 3.23, while the mean
decency of the responses generated by GPT-4 is 3.01. The consistency is
not too different between the two versions, with GPT-3.5 having an
average of 3.61, while GPT-4 has an average of 3.65. GPT-4 appears to
have fewer responses that are completely identical
(Figure~\ref{fig-version-2}).

\begin{figure}

\begin{minipage}[t]{0.50\linewidth}

{\centering 

\raisebox{-\height}{

\includegraphics{paper_files/figure-pdf/fig-version-1.pdf}

}

}

\subcaption{\label{fig-version-1}Decency}
\end{minipage}%
%
\begin{minipage}[t]{0.50\linewidth}

{\centering 

\raisebox{-\height}{

\includegraphics{paper_files/figure-pdf/fig-version-2.pdf}

}

}

\subcaption{\label{fig-version-2}Consistency}
\end{minipage}%

\caption{\label{fig-version}How decency and consistency change based on
whether GPT-3.5 or GPT-4 is used}

\end{figure}

Figure~\ref{fig-prompt} examines how decency and consistency differ
based on which prompt is used. The prompts differ by how much
information is provided. The least informative prompt, ``Name'',
essentially consists of just providing the LLM with the names of the
columns that are expected to be in the dataset. The next most
informative prompt, ``Describe'', adds a detailed description of what we
expect of the observations. ``Simulate'' adds that we expect the LLM to
first simulate a dataset based on that description, before generating
the expectations. And finally, the most informative prompt, ``Example'',
provides a snapshot of the dataset, consisting of the relevant variables
and ten observations.

There appears to be considerable difference in terms of how the prompts
are associated with decency. In particular, ``Name'' is never associated
with a 5/5 rating (Figure~\ref{fig-prompt-1}). Surprisingly, however,
the most informative prompt, ``Example'', is also never associated with
a 5/5 rating. Instead it is ``Describe'' and ``Simulate'' that tend to
be associated with better decency ratings. This is reflected in the
averages, which are 2.65, 3.46, 3.44, and 2.94 for ``Name'',
``Describe'', ``Simulate'', and ``Example'', respectively.

The pattern is not as clear when it comes to consistency
(Figure~\ref{fig-prompt-2}). All four have similar averages, at 3.71,
3.75, 3.48, and 3.58 for ``Name'', ``Describe'', ``Simulate'', and
``Example'', respectively. That said, it is clear that a wider variety
of responses (as denoted by lower consistency ratings), are rarely seen
for ``Name'' and ``Describe''.

\begin{figure}

\begin{minipage}[t]{0.50\linewidth}

{\centering 

\raisebox{-\height}{

\includegraphics{paper_files/figure-pdf/fig-prompt-1.pdf}

}

}

\subcaption{\label{fig-prompt-1}Decency}
\end{minipage}%
%
\begin{minipage}[t]{0.50\linewidth}

{\centering 

\raisebox{-\height}{

\includegraphics{paper_files/figure-pdf/fig-prompt-2.pdf}

}

}

\subcaption{\label{fig-prompt-2}Consistency}
\end{minipage}%

\caption{\label{fig-prompt}How decency and consistency change based on
the type of prompt used}

\end{figure}

Temperature is a parameter that varies from 0 to 1, that we can use to
manipulate how random the LLM is. At high temperatures, the LLM will
produce a wider variety of responses. At lower temperatures it will
focus on the single most likely response. Higher temperature should be
associated with a wider variety of LLM responses.

Figure~\ref{fig-temperature} examines how decency and consistency differ
based on which of the four temperature values we consider---0, 0.4, 0.6,
1---is used. There appears to be limited difference in terms of how
different temperature values are associated with decency
(Figure~\ref{fig-temperature-1}). They all have similar mean values at
3.10, 3.25, 2.98, and 3.15 for temperature values of 0, 0.4, 0.6, and 1,
respectively.

In contrast, as expected temperature has an effect on consistency.
Temperature values of 0 are associated with ratings of high consistency,
and higher temperatures are associated with lower consistency
(Figure~\ref{fig-temperature-2}). Their means differ considerably, with
4.77, 3.75, 3.31, and 2.69 for temperature values of 0, 0.4, 0.6, and 1,
respectively.

\begin{figure}

\begin{minipage}[t]{0.50\linewidth}

{\centering 

\raisebox{-\height}{

\includegraphics{paper_files/figure-pdf/fig-temperature-1.pdf}

}

}

\subcaption{\label{fig-temperature-1}Decency}
\end{minipage}%
%
\begin{minipage}[t]{0.50\linewidth}

{\centering 

\raisebox{-\height}{

\includegraphics{paper_files/figure-pdf/fig-temperature-2.pdf}

}

}

\subcaption{\label{fig-temperature-2}Consistency}
\end{minipage}%

\caption{\label{fig-temperature}How decency and consistency change based
on temperature}

\end{figure}

Role is an aspect of a prompt that is provided to the LLM before the
main prompt content. We considered two different roles, one that
positioned the LLM as a helpful assistant, and the other that positioned
the LLM as an experienced data scientist (Figure~\ref{fig-role}). We
were expecting that the expert role would result in better code, but
there was no obvious difference in terms of decency
(Figure~\ref{fig-role-1}) or consistency (Figure~\ref{fig-role-2}).
Their means did not differ by much in either case.

\begin{figure}

\begin{minipage}[t]{0.50\linewidth}

{\centering 

\raisebox{-\height}{

\includegraphics{paper_files/figure-pdf/fig-role-1.pdf}

}

}

\subcaption{\label{fig-role-1}Decency}
\end{minipage}%
%
\begin{minipage}[t]{0.50\linewidth}

{\centering 

\raisebox{-\height}{

\includegraphics{paper_files/figure-pdf/fig-role-2.pdf}

}

}

\subcaption{\label{fig-role-2}Consistency}
\end{minipage}%

\caption{\label{fig-role}How decency and consistency change based on
whether role is `Helpful' or `Expert'}

\end{figure}

Learning mode refers to the number of examples provided to the LLM as
part of the prompt. Zero-shot learning means that no examples are
provided, while one-shot and few-shot learning refer to one- and a few-
examples being provided, respectively. Although the advantage of LLMs
such as GPT-3.5 and GPT-4 is that they typically do well with zero-shot
learning, we would expect that they will do better with one-shot and
few-shot.

We find substantial differences, especially when moving away from
zero-shot learning (Figure~\ref{fig-shot}). In particular, we see that
decency of 1/5 is dominated by zero-shot, while zero-shot is
under-represented in 5/5 (Figure~\ref{fig-shot-1}). This is also
reflected in the mean decency which for zero-shot is 2.83, while for
one- and few-shot learning is 3.34 and 3.19, respectively.

We see this pattern in consistency as well. For instance, zero-shot
learning is over-represented in the least consistent responses, both 1/5
and 2/5 (Figure~\ref{fig-shot-2}). And the mean level of consistency is
lower for zero-shot learning, at 3.45, compared with single- and
few-shot, at 3.77 and 3.67, respectively.

\begin{figure}

\begin{minipage}[t]{0.50\linewidth}

{\centering 

\raisebox{-\height}{

\includegraphics{paper_files/figure-pdf/fig-shot-1.pdf}

}

}

\subcaption{\label{fig-shot-1}Decency}
\end{minipage}%
%
\begin{minipage}[t]{0.50\linewidth}

{\centering 

\raisebox{-\height}{

\includegraphics{paper_files/figure-pdf/fig-shot-2.pdf}

}

}

\subcaption{\label{fig-shot-2}Consistency}
\end{minipage}%

\caption{\label{fig-shot}How decency and consistency change based on
zero-, one-, and few-shot learning}

\end{figure}

\hypertarget{sec-model}{%
\section{Model}\label{sec-model}}

Consistency and decency, our dependent variables, are ordered,
categorical, outcomes. As such we model the rating \(Y\), which is an
ordinal outcome with \(J = 5\) possible categories, using ordered
logistic regression. Such a model works by estimating the probability
that the outcome is less than or equal to some specific category, which
in this case is the coded rank of that response (i.e.~1, 2, 3, 4, 5).
Models for consistency and decency are estimated separately. In latent
variable formulation, we assume that the observed category \(y\) can be
related to an underlying continuous latent variable, \(y^*\), through a
series of cutpoints:

\[
y=\left\{\begin{array}{ll}
1 & \text { if } y^*<\zeta_1 \\
2 & \text { if } \zeta_1 \leq y^* \\
\vdots & \\
J & \text { if } \zeta_{I-1}<1
\end{array}\right.
\]

where \(\zeta\) is a vector of cutpoints. The latent variable \(y^*\) is
then assumed to have a logistic distribution and is modeled as a linear
function of covariates. In our case, we are interested in exploring the
relationships that consistency and decency have with model, prompt,
temperature, role, and learning mode: \[
y^* = \beta_1 \cdot \mbox{version}_i + \beta_2 \cdot \mbox{prompt}_i + \beta_3 \cdot \mbox{temperature}_i + \beta_4 \cdot \mbox{role}_i + \beta_5 \cdot \mbox{shot}_i
\]

In terms of our predictors, version is a binary variable as to whether
we are using GPT-3.5 or GPT-4. We expect that GPT-4 will do better in
terms of both decency and consistency than GPT-3.5. Prompt can be one of
four values: ``Name'', ``Describe'', ``Simulate'', and ``Example''. We
expect that ``Simulate'' and ``Example'' will be associated with higher
decency than ``Name'' and ``Describe''. However, we expect the opposite
relationship with consistency. This is because we expect that what will
increase decency will be the more specific guidance provided by the
``Simulate'' and ``Example'' prompts, which should also increase the
consistency. Temperature can take one of four values: 0, 0.4, 0.6 and 1.
We expect that higher temperature values will be associated with less
consistency. However, it is difficult to anticipate how temperature
should be related to decency. Role can be one of two values:
``Helpful'', or ``Expert'', while learning mode can be one of three:
``zero'', ``one'', or ``few''. In the case of both role and learning
mode, we expect that the ``Expert'' role, and one- and few-shot learning
will be associated with higher decency, and consistency.

We fit this model, separately, for each of consistency and decency, in a
Bayesian framework using the package \texttt{rstanarm} (Goodrich et al.
2023) and the R statistical programming language (R Core Team 2023).
This computational process requires priors to be placed on \(R^2\), the
proportion of variance in the outcome that is attributable to the
coefficients in a linear model, and the vector of cutpoints \(\zeta\).
For the prior on \(R^2\), we follow Gelman, Hill, and Vehtari (2020,
276) and assume the mean is 0.3. For the vector cutpoints \(\zeta\), we
use an uninformative prior of \(\text{Dirichlet}(1)\). Gabry and
Goodrich (2020) provides more information about fitting this type of
model using \texttt{rstanarm}.

Diagnostics are provided in Section~\ref{sec-diagnostics}.

\hypertarget{sec-results}{%
\section{Results}\label{sec-results}}

In total we considered 192 observations, which is 96 for each LLM
considered. The estimates from our models are shown in
Table~\ref{tbl-modelsummaryestimates} and
Figure~\ref{fig-modelsummaryestimates}, which were both produced with
\texttt{modelsummary} (Arel-Bundock 2022).

\hypertarget{tbl-modelsummaryestimates}{}
\begin{table}
\caption{\label{tbl-modelsummaryestimates}Exploring the relationships that consistency and decency have with
model, prompt, temperature, role, and learning mode }\tabularnewline

\centering
\begin{tabular}[t]{lcc}
\toprule
  & Consistency & Decency\\
\midrule
VersionGPT4 & \num{-0.03} & \num{-0.29}\\
 & (\num{0.25}) & (\num{0.24})\\
Prompt\_nDescribe & \num{0.12} & \num{1.30}\\
 & (\num{0.35}) & (\num{0.36})\\
Prompt\_nSimulate & \num{-0.31} & \num{1.25}\\
 & (\num{0.36}) & (\num{0.34})\\
Prompt\_nExample & \num{-0.15} & \num{0.46}\\
 & (\num{0.35}) & (\num{0.31})\\
Temperature0.4 & \num{-2.19} & \num{0.13}\\
 & (\num{0.41}) & (\num{0.31})\\
Temperature0.6 & \num{-2.87} & \num{-0.11}\\
 & (\num{0.45}) & (\num{0.32})\\
Temperature1 & \num{-3.83} & \num{0.06}\\
 & (\num{0.50}) & (\num{0.33})\\
Role\_nExpert & \num{0.02} & \num{0.09}\\
 & (\num{0.25}) & (\num{0.23})\\
Shot\_nOne & \num{0.55} & \num{0.76}\\
 & (\num{0.30}) & (\num{0.28})\\
Shot\_nFew & \num{0.35} & \num{0.56}\\
 & (\num{0.31}) & (\num{0.29})\\
1|2 & \num{-6.43} & \num{-1.76}\\
 & (\num{0.68}) & (\num{0.44})\\
2|3 & \num{-3.97} & \num{-0.32}\\
 & (\num{0.53}) & (\num{0.40})\\
3|4 & \num{-2.61} & \num{1.82}\\
 & (\num{0.49}) & (\num{0.43})\\
4|5 & \num{-0.61} & \num{3.87}\\
 & (\num{0.46}) & (\num{0.49})\\
\midrule
Num.Obs. & \num{192} & \num{192}\\
ELPD & \num{-227.6} & \num{-255.5}\\
ELPD s.e. & \num{8.9} & \num{9.3}\\
LOOIC & \num{455.1} & \num{511.0}\\
LOOIC s.e. & \num{17.8} & \num{18.6}\\
WAIC & \num{455.1} & \num{510.9}\\
\bottomrule
\end{tabular}
\end{table}

\begin{figure}

{\centering \includegraphics{paper_files/figure-pdf/fig-modelsummaryestimates-1.pdf}

}

\caption{\label{fig-modelsummaryestimates}Exploring the relationships
that consistency and decency have with model, prompt, temperature, role,
and learning mode}

\end{figure}

Table~\ref{tbl-modelsummaryestimates} shows the coefficient estimate,
with the mad statistic (mean absolute deviation) in brackets. In
Figure~\ref{fig-modelsummaryestimates}, the dot represents the
coefficient estimate, while the lines are 90 per cent credibility
intervals.

The intercepts, 1\textbar2, 2\textbar3, 3\textbar4, and 4\textbar5,
shown in Table~\ref{tbl-modelsummaryestimates} reflect cutpoints, where
a rating goes from one category to the next (e.g.~from 1 to 2, etc).

The models do not identify a substantial difference between GPT-3.5 and
GPT-4 in terms of consistency, and, surprisingly, even a somewhat
negative association in terms of decency.

Our models do not identify much difference between the different prompt
types in terms of consistency. However, they do find that ``Describe''
and ``Simulate'' are associated with increased decency, as is
``Example'', although surprisingly to a lesser extent.

As expected, increases in temperature are strongly associated with less
consistency. However, we are unable to identify much of a relationship
between temperature and decency.

The models struggle to find any association between which role is used
in terms of either consistency or decency. That is, priming the prompt
with either ``You are a helpful assistant.'' or ``You are a
highly-trained, experienced, data scientist who is an expert at writing
readable, correct, Python code.'' did not seem to change much in terms
of the coding of the responses.

Finally, one- and few-shot learning is associated with increased decency
and slightly increased consistency over zero-shot prompts. That said,
the magnitude of this relationship is not overly large.

It is worth noting that the rating of the coherence and decency was a
human endeavor, with all the bias that brings. In terms of decency,
sometimes the LLMs simply did not generate code, instead generating
natural language text about what tests could look like. Other times the
LLMs generated a lot of very similar tests, for instance going through
many possible options testing for class or non-null results, at the
expense of developing a broadly useful suite. Finally, sometimes the
LLMs would generate only one or two tests, or generate tests without any
comments or other explanations. With regard to coherence, it was often
the case that a low coherence rating was due to one of the five
responses being especially difference. It was rare that the LLM just
entirely misinterpreted the prompt, but compared with a human coder, who
brings a broad context to writing tests, the LLMs rarely took into
account the entire context that we would like.

\hypertarget{sec-discussion}{%
\section{Discussion}\label{sec-discussion}}

\hypertarget{on-the-importance-and-challenges-of-writing-validation-tests}{%
\subsection{On the importance and challenges of writing validation
tests}\label{on-the-importance-and-challenges-of-writing-validation-tests}}

Data validation is a cornerstone of high quality data science workflows.
Broadly, data validation is centered around establishing the credibility
of individual data values, the coherence of data with itself, and the
consistency of data with relevant external sources (Alexander 2023).
Data quality --- or a lack thereof --- can have fundamental downstream
effects on the rest of the scientific process. If the data being
supplied to a statistical model is incorrect in some way (e.g., a
categorical variable is stored as numeric), this will have significant
implications for subsequent results and conclusions (Alexander 2023;
Hynes, Sculley, and Terry 2017; Breck et al. 2019; Zhang et al. 2023;
Gao, Xie, and Tao 2016). A powerful tool to implement data validation in
practice is automated testing, which intuits data expectations from
format, column name, or other meta information and converts those
expectations to code (Alexander 2023; Zhang et al. 2023). Automated
validation promotes trustworthiness and transparency. By developing a
suite of tests to validate that particular characteristics of a dataset
hold true, data scientists uncover fundamental assumptions that underpin
their work and equip users with confidence that those assumptions are
met in practice. The value of data validation testing is not limited to
particular domains --- it is a fundamental component of a reliable data
science workflow, regardless of the data at hand (Alexander 2023; Gao,
Xie, and Tao 2016).

There are a number of challenges that accompany the development and
implementation of data validation tests. These challenges are
particularly important to consider when thinking about data validation
efforts across research domains and levels of data science expertise.

First, implementing automated tests requires the ability to develop
code. There are a number of testing-centered libraries in various
programming languages. \texttt{Great\ Expectations} --- the Python
library employed by Katz and Moore (2023) --- contains various data
testing functions and has a data assistant functionality which
programmatically explores data to develop tests based on observed
characteristics (Great Expectations 2023). \texttt{pointblank} is an R
package which offers predefined automated data testing functionality
(Iannone and Vargas 2023) and IBM SPSS statistics can perform data
validation checks by identifying unexpected or invalid entries in a
dataset (IBM Knowledge Center 2023). With a focus on data validation for
machine learning (ML), Hynes, Sculley, and Terry (2017) developed a data
linter tool that detects possible inconsistencies in the data and
suggests appropriate transformations for a specific model type. Further,
Breck et al. (2019) executed an automated validation ML platform at
Google to address concerns about the downstream effects of messy data in
model training. While various data testing software have been developed
and shared, many of these have limited functionality, and the
implementation of a test suite may require significant data processing
work and necessarily requires programming knowledge (Alexander 2023;
Zhang et al. 2023). From their work on data validation for the IJF Katz
and Moore (2023) conclude that data tests should not be compromised just
to be compatible with predefined test functions that exist in a software
package. In order to develop the most valuable suite of tests, it may be
necessary to create new variables in a dataset, merge multiple datasets,
or create tests which are outside the scope of existing software
frameworks, for example. This presents a challenge for individuals who
are not trained in or comfortable with computer programming, but wish to
programmatically test their data.

In addition, the development of a comprehensive, bespoke, and accurate
data test suite can rarely be done by one data scientist alone. As
advocated by Katz and Moore (2023), domain expertise is fundamental to
the creation and development of high-quality data tests. For instance,
without incorporating the knowledge that donation dates are only
collected in three jurisdictions (British Columbia, Ontario and Federal)
into their code, Katz and Moore (2023)'s test for missingness for that
variable would have produced inflated failure rates. This is knowledge
that was gained through collaborative efforts with IJF team members who
were involved in dataset construction. For an individual working on an
unfamiliar dataset, seeking out domain knowledge and gaining a thorough
understanding of that data is a non-trivial task in and of itself.

Finally, the process of designing, developing, and deploying data
validation tests is iterative in nature and takes a great deal of time
(Katz and Moore 2023). For researchers or journalists who are eager to
publish their findings, for instance, there may not be enough incentive
or resources for them to do this time-consuming validation work. This
challenge is only amplified if an individual was not involved in the
dataset construction process, does not have access to vital domain
expertise, or does not have programming experience.

While data validation is incredibly valuable for the production of
transparent and trustworthy research, it comes with many challenges
which hinder its accessibility. This work presents an opportunity to
advocate for an alternative approach to data validation through the use
of LLMs. Though there has been a great deal of work done to develop more
accessible programming tools for data validation, these tools do not
entirely mitigate the challenges which accompany the process, namely the
need for programming knowledge, domain expertise, and a great deal of
time. Our findings illustrate the potential for LLMs to rapidly produce
usable data validation tests, written in code, with limited information
--- a promising application to support the promotion of more
trustworthy, transparent data workflows across domains.

\hypertarget{on-the-use-of-llms-in-data-science}{%
\subsection{On the use of LLMs in data
science}\label{on-the-use-of-llms-in-data-science}}

The contributions of these experiments and this paper are twofold.
Firstly, this work adds to the emerging area of inquiry of prompt
engineering for large language models. In doing so, our prompts may
serve as examples -- both good and bad -- for other scientists hoping to
implement LLMs into their research and data workflows. Secondly, this
work contributes to the academic study of LLMs for the development of
data science methodologies. As these tools become more powerful and the
precision in our ability to specify their outputs improves, they will
become tools to reduce the time and resource burden of writing software
and tests.

Our work is consistent with previous literature demonstrating that LLM
performance on complex, user-defined tasks is sensitive to prompt
engineering. Prompting is a brittle process in which minor alterations
can lead to large fluctuations in performance (Arora et al. 2022; Zhao
et al. 2021), however with OpenAI's LLMs in particular, GPT-4 (Giorgi et
al. 2023) seems less susceptible to this than ChatGPT (GPT-3.5). In
general, previous research has demonstrated that when using ChatGPT,
prompts should provide context (White et al. 2023; Clavié et al. 2023),
be specific (White et al. 2023; Yong et al. 2023; Shieh, n.d.), define a
persona or role (White et al. 2023; Clavié et al. 2023), break complex
reasoning into smaller steps (White et al. 2023; Henrickson and
Meroño-Peñuela 2023), and provide example responses (Brown et al. 2020;
Arora et al. 2022; Shieh, n.d.) to improve performance on a variety of
tasks, including classifying job types (Clavié et al. 2023), generating
images for construction defect detection (Yong et al. 2023), and
generating hermeneutically valuable text (Henrickson and Meroño-Peñuela
2023).

These observations are predominantly borne out in our experiments. For
instance, the more specified prompt (``Describe'') and the prompt
eliciting step-by-step thinking (``Simulate'') both show improvements
over the simplest prompt (``Name''). However, improvements made by
specificity break down when example data is provided (``Example''). We
hypothesize that this is due to ChatGPT and GPT-4's poor performance on
numerical and structured data in comparison to more domain-specific LLMs
such as BloombergGPT which has been trained on numeric, structured
financial data and utilizes a tokenizer more suited to these data (Wu et
al. 2023). We do not see significant changes in performance when two
different roles are specified to the LLM, which contradicts previous
literature (White et al. 2023; Clavié et al. 2023). Finally, our
significant improvements moving from zero- to one-shot and marginal to
no improvements from one- to few-shot align with existing literature
(Giorgi et al. 2023). Some studies have demonstrated ChatGPT and GPT-4's
difficulties in responding to queries in the desired format during
zero-shot structured tasks, including named entity recognition (Hu et
al. 2023) and multiple choice question (Clavié et al. 2023). Giorgi et
al. (2023) thus attribute performance improvements in one- and few-shot
prompts to the in-context examples' guidance as to the desired output
structure. Upon inspection of the outputs from zero-, one-, and few-shot
prompts, we believe that this same effect exists for our prompts.

Our work additionally contributes to the growing body of work using LLMs
to automate portions of the data science workflow. LLMs have already
been integrated into software tools such as GitHub's Co-Pilot ({``{Your
AI pair programmer},''} n.d.) and AutoGPT ({``{AutoGPT},''} n.d.) and
IDEs such as IntelliJ (Krochmalski 2104) and VSCode (Dias 2023) to
produce code in a variety of languages. These tools are and will
continue to assist data scientists to accelerate, resolve bugs in,
document, and optimize their code in all parts of the data science
workflow. LLMs have also demonstrated potential to automate other
onerous or tedious tasks in data science pipelines. These include
generating synthetic text data (Chung, Kamar, and Amershi 2023; Yu et
al. 2023), improving search-based software testing (Lemieux et al.
2023), generating unit tests from natural language in a variety of
programming languages and contexts (Lahiri et al. 2022; Schafer et al.
2023), generating property-based tests from API documentation (Vikram,
Lemieux, and Padhye 2023), and conducting exploratory data analysis (Ma
et al. 2023). As LLM prompting and fine-tuning methods improve, they
will reduce technical barriers to conducting transparent, accurate,
reproducible analysis, enabling data scientists to be more productive
and put their energy towards analysis. higher-order reasoning, and
sophisticated inference. The work described in this paper towards
automating data validation using LLMs has the capacity to further this
evolution.

\hypertarget{limitations}{%
\subsection{Limitations}\label{limitations}}

There are a variety of limitations of this study. The most notable is
that there was only one coder of the LLMs responses. While this should
ensure internal consistency, it is possible that some of the coding,
especially that for decency, would have benefited from an additional
coder, and their responses could then have been averaged. The study also
only considers one, reasonably specific, setting. Expanding our approach
to consider a variety of settings could be especially beneficial.

A limitation of our work on prompt engineering (and indeed of prompt
engineering generally) is that we are unable to provide fully reliable
explanations for improvements in performance. Though we can make
convincing speculations, these are ultimately educated guesses. Previous
studies have shown that in addition to being brittle to precise
semantics (Arora et al. 2022; Zhao et al. 2021), prompts show improved
performance from additional text which provides no new information
(Henrickson and Meroño-Peñuela 2023) such as naming the model (e.g.,
``You are Frederick, a helpful AI''), emulating the chatbot saying that
it understands the instructions, asking it to output the ``right
conclusion'', and providing positive feedback (Clavié et al. 2023).
Despite the opacity of the emergent properties of LLMs, we may be able
to develop greater understanding of the impacts of prompting choices by
testing these prompts on a greater diversity of datasets from across
disciplines and contexts.

Our current experiments have been run in a very limited context ---
namely one Canadian political donations dataset. We were also focused on
writing tests within the \texttt{Great\ Expectations} framework. It may
be that testing these prompts in a variety of contexts, and also
considering potential avenues of exploration beyond
\texttt{Great\ Expectations}, will provide evidence as to whether our
prompting strategies are generalizable across datasets in different
formats, contexts, and technical regimes. Prompts can be conceptualized
as knowledge transfer methods analogous to software patterns which are
intended to provide reusable solutions to common problems (White et al.
2023). This aligns with the goal of automated data validation to reduce
the need for domain expertise and resources dedicated to data
validation. Thus, in order to improve the explainability of our methods
and to ensure our method is generalizable, we must validate that the
prompts that are effective on this dataset show similar performance on
other datasets from different contexts and in different formats.

\newpage

\hypertarget{appendix}{%
\section{Appendix}\label{appendix}}

\hypertarget{sec-diagnostics}{%
\subsection{Model diagnostics}\label{sec-diagnostics}}

\hypertarget{consistency-model}{%
\subsubsection{Consistency model}\label{consistency-model}}

\begin{figure}

{\centering \includegraphics{paper_files/figure-pdf/unnamed-chunk-14-1.pdf}

}

\caption{Consistency model trace plot}

\end{figure}

\begin{figure}

{\centering \includegraphics{paper_files/figure-pdf/unnamed-chunk-15-1.pdf}

}

\caption{Consistency model rhat plot}

\end{figure}

\begin{figure}

{\centering \includegraphics{paper_files/figure-pdf/unnamed-chunk-16-1.pdf}

}

\caption{Consistency model loo plot}

\end{figure}

\begin{figure}

{\centering \includegraphics{paper_files/figure-pdf/unnamed-chunk-17-1.pdf}

}

\caption{Consistency model posterior predictive check}

\end{figure}

\begin{figure}

{\centering \includegraphics{paper_files/figure-pdf/unnamed-chunk-18-1.pdf}

}

\caption{Consistency model posterior vs prior}

\end{figure}

\newpage

\hypertarget{decency-model}{%
\subsubsection{Decency model}\label{decency-model}}

\begin{figure}

{\centering \includegraphics{paper_files/figure-pdf/unnamed-chunk-19-1.pdf}

}

\caption{Decency model trace plot}

\end{figure}

\begin{figure}

{\centering \includegraphics{paper_files/figure-pdf/unnamed-chunk-20-1.pdf}

}

\caption{Decency model rhat plot}

\end{figure}

\begin{figure}

{\centering \includegraphics{paper_files/figure-pdf/unnamed-chunk-21-1.pdf}

}

\caption{Decency model loo plot}

\end{figure}

\begin{figure}

{\centering \includegraphics{paper_files/figure-pdf/unnamed-chunk-22-1.pdf}

}

\caption{Decency model posterior predictive check}

\end{figure}

\begin{figure}

{\centering \includegraphics{paper_files/figure-pdf/unnamed-chunk-23-1.pdf}

}

\caption{Decency model posterior vs prior}

\end{figure}

\newpage

\hypertarget{references}{%
\section*{References}\label{references}}
\addcontentsline{toc}{section}{References}

\hypertarget{refs}{}
\begin{CSLReferences}{1}{0}
\leavevmode\vadjust pre{\hypertarget{ref-alexander2023}{}}%
Alexander, Rohan. 2023. \emph{{Telling Stories with Data: With
Applications in R}}. CRC Press.

\leavevmode\vadjust pre{\hypertarget{ref-modelsummary}{}}%
Arel-Bundock, Vincent. 2022. {``{modelsummary}: Data and Model Summaries
in {R}.''} \emph{Journal of Statistical Software} 103 (1): 1--23.
\url{https://doi.org/10.18637/jss.v103.i01}.

\leavevmode\vadjust pre{\hypertarget{ref-arora2022}{}}%
Arora, Simran, Avanika Narayan, Mayee F. Chen, Laurel Orr, Neel Guha,
Kush Bhatia, Ines Chami, Frederic Sala, and Christopher Ré. 2022.
{``{Ask Me Anything: A simple strategy for prompting language
models}.''} \url{https://arxiv.org/abs/2210.02441}.

\leavevmode\vadjust pre{\hypertarget{ref-autogpt}{}}%
{``{AutoGPT}.''} n.d. \url{https://autogpt.net/}; AutoGPT.

\leavevmode\vadjust pre{\hypertarget{ref-breck2019}{}}%
Breck, Eric, Neoklis Polyzotis, Sudip Roy, Steven Euijong Whang, and
Martin Zinkevich. 2019. {``{Data validation for machine learning}.''} In
\emph{Proceedings of Machine Learning and Systems 2019, MLSys 2019},
edited by A. Talwalkar, V. Smith, and M. Zaharia, 334--47. Stanford, CA,
USA: mlsys.org.
\url{https://mlsys.org/Conferences/2019/doc/2019/167.pdf}.

\leavevmode\vadjust pre{\hypertarget{ref-brown2020}{}}%
Brown, Tom B., Benjamin Mann, Nick Ryder, Melanie Subbiah, Jared Kaplan,
Prafulla Dhariwal, Arvind Neelakantan, et al. 2020. {``{Language Models
are Few-Shot Learners}.''} \emph{{CoRR}} abs/2005.14165.
\url{https://arxiv.org/abs/2005.14165}.

\leavevmode\vadjust pre{\hypertarget{ref-chung2023}{}}%
Chung, John, Ece Kamar, and Saleema Amershi. 2023. {``{Increasing
Diversity While Maintaining Accuracy: Text Data Generation with Large
Language Models and Human Interventions}.''} In \emph{Proceedings of the
61st Annual Meeting of the Association for Computational Linguistics
(Volume 1: Long Papers)}, 575--93. Toronto, Canada: Association for
Computational Linguistics.
\url{https://doi.org/10.18653/v1/2023.acl-long.34}.

\leavevmode\vadjust pre{\hypertarget{ref-clavie2023}{}}%
Clavié, Benjamin, Alexandru Ciceu, Frederick Naylor, Guillaume Soulié,
and Thomas Brightwell. 2023. {``{Large Language Models in~the~Workplace:
A Case Study on~Prompt Engineering for~Job Type Classification}.''} In
\emph{Natural Language Processing and Information Systems}, edited by
Elisabeth Métais, Farid Meziane, Vijayan Sugumaran, Warren Manning, and
Stephan Reiff-Marganiec, 3--17. Cham: Springer Nature Switzerland.

\leavevmode\vadjust pre{\hypertarget{ref-vscode}{}}%
Dias, Chris. 2023. {``{Visual Studio Code and GitHub Copilot}.''}
\url{https://code.visualstudio.com/blogs/2023/03/30/vscode-copilot};
Microsoft.

\leavevmode\vadjust pre{\hypertarget{ref-stanpolrvignette}{}}%
Gabry, Jonah, and Ben Goodrich. 2020. \emph{{Estimating Ordinal
Regression Models with rstanarm}}.
\url{https://mc-stan.org/rstanarm/articles/polr.html}.

\leavevmode\vadjust pre{\hypertarget{ref-gao2016}{}}%
Gao, Jerry, Chunli Xie, and Chuanqi Tao. 2016. {``{Big data validation
and quality assurance--issuses, challenges, and needs}.''} In \emph{2016
IEEE Symposium on Service-Oriented System Engineering (SOSE)}, 433--41.
IEEE.

\leavevmode\vadjust pre{\hypertarget{ref-gelmanhillvehtari2020}{}}%
Gelman, Andrew, Jennifer Hill, and Aki Vehtari. 2020. \emph{Regression
and Other Stories}. Cambridge University Press.
\url{https://avehtari.github.io/ROS-Examples/}.

\leavevmode\vadjust pre{\hypertarget{ref-giorgi2023}{}}%
Giorgi, John, Augustin Toma, Ronald Xie, Sondra Chen, Kevin An, Grace
Zheng, and Bo Wang. 2023. {``{WangLab at MEDIQA-Chat 2023: Clinical Note
Generation from Doctor-Patient Conversations using Large Language
Models}.''} In \emph{Proceedings of the 5th Clinical Natural Language
Processing Workshop}, 323--34. Toronto, Canada: Association for
Computational Linguistics.
\url{https://doi.org/10.18653/v1/2023.clinicalnlp-1.36}.

\leavevmode\vadjust pre{\hypertarget{ref-rstanarm}{}}%
Goodrich, Ben, Jonah Gabry, Imad Ali, and Sam Brilleman. 2023.
{``{rstanarm: Bayesian applied regression modeling via Stan.}''}
\url{https://mc-stan.org/rstanarm/}.

\leavevmode\vadjust pre{\hypertarget{ref-gx23}{}}%
Great Expectations. 2023. {``{Create an Expectation Suite with the
Onboarding Data Assistant}.''} 2023.
\url{https://docs.greatexpectations.io/docs/guides/expectations/data_assistants/how_to_create_an_expectation_suite_with_the_onboarding_data_assistant/}.

\leavevmode\vadjust pre{\hypertarget{ref-henrickson2023}{}}%
Henrickson, Leah, and Albert Meroño-Peñuela. 2023. {``{Prompting
Meaning: A Hermeneutic Approach to Optimising Prompt Engineering with
ChatGPT}.''} \emph{{AI \& Society}}, September.
\url{https://doi.org/10.1007/s00146-023-01752-8}.

\leavevmode\vadjust pre{\hypertarget{ref-hu2023}{}}%
Hu, Yan, Iqra Ameer, Xu Zuo, Xueqing Peng, Yujia Zhou, Zehan Li, Yiming
Li, Jianfu Li, Xiaoqian Jiang, and Hua Xu. 2023. {``{Zero-shot Clinical
Entity Recognition using ChatGPT}.''}
\url{https://arxiv.org/abs/2303.16416}.

\leavevmode\vadjust pre{\hypertarget{ref-hynes17}{}}%
Hynes, Nick, D. Sculley, and Michael Terry. 2017. {``{The Data Linter:
Lightweight Automated Sanity Checking for ML Data Sets}.''} In.
\url{http://learningsys.org/nips17/assets/papers/paper_19.pdf}.

\leavevmode\vadjust pre{\hypertarget{ref-pointblank}{}}%
Iannone, Richard, and Mauricio Vargas. 2023. \emph{{pointblank: Data
Validation and Organization of Metadata for Local and Remote Tables}}.
\url{https://CRAN.R-project.org/package=pointblank}.

\leavevmode\vadjust pre{\hypertarget{ref-IBM23}{}}%
IBM Knowledge Center. 2023. {``{SPSS Statistics Documentation: Validate
Data}.''} 2023.
\url{https://www.ibm.com/docs/en/spss-statistics/29.0.0?topic=preparation-validate-data}.

\leavevmode\vadjust pre{\hypertarget{ref-katzandmoore}{}}%
Katz, Lindsay, and Callandra Moore. 2023. {``{Implementing Automated
Data Validation for Canadian Political Datasets}.''}
\url{https://doi.org/10.48550/arXiv.2309.12886}.

\leavevmode\vadjust pre{\hypertarget{ref-krochmalski2014}{}}%
Krochmalski, Jaroslaw. 2104. \emph{{IntelliJ IDEA Essentials}}. Packt
Publishing Ltd.

\leavevmode\vadjust pre{\hypertarget{ref-lahiri2022}{}}%
Lahiri, Shuvendu K., Aaditya Naik, Georgios Sakkas, Piali Choudhury,
Curtis von Veh, Madanlal Musuvathi, Jeevana Priya Inala, Chenglong Wang,
and Jianfeng Gao. 2022. {``{Interactive code generation via test-driven
user-intent formalization}.''} \emph{{arXiv}}.

\leavevmode\vadjust pre{\hypertarget{ref-lemieux2023}{}}%
Lemieux, Caroline, Jeevana Priya Inala, Shuvendu K Lahiri, and
Siddhartha Sen. 2023. {``{Codamosa: Escaping coverage plateaus in test
generation with pre-trained large language models}.''} In \emph{45th
International Conference on Software Engineering}. ICSE.

\leavevmode\vadjust pre{\hypertarget{ref-ma2023}{}}%
Ma, Pingchuan, Rui Ding, Shuai Wang, Shi Han, and Dongmei Zhang. 2023.
{``{Demonstration of InsightPilot: An LLM-Empowered Automated Data
Exploration System}.''} \url{https://arxiv.org/abs/2304.00477}.

\leavevmode\vadjust pre{\hypertarget{ref-ouyang2022}{}}%
Ouyang, Long, Jeff Wu, Xu Jiang, Diogo Almeida, Carroll L. Wainwright,
Pamela Mishkin, Chong Zhang, et al. 2023. {``{Training language models
to follow instructions with human feedback}.''} NeurIPS.

\leavevmode\vadjust pre{\hypertarget{ref-citeR}{}}%
R Core Team. 2023. \emph{R: A Language and Environment for Statistical
Computing}. Vienna, Austria: R Foundation for Statistical Computing.
\url{https://www.R-project.org/}.

\leavevmode\vadjust pre{\hypertarget{ref-schafer2023}{}}%
Schafer, Max, Sarah Nadi, Aryaz Eghbali, and Frank Tip. 2023.
{``{Adaptive test generation using a large language model}.''}
\url{https://arxiv.org/abs/2302.06527v3}.

\leavevmode\vadjust pre{\hypertarget{ref-openaiprompt}{}}%
Shieh, Jessica. n.d. {``{Best practices for prompt engineering with
OpenAI API}.''}
\url{https://help.openai.com/en/articles/6654000-best-practices-for-prompt-engineering-with-openai-api};
{OpenAI}.

\leavevmode\vadjust pre{\hypertarget{ref-ijfmethods}{}}%
The IJF. 2023. {``{Donations Methodology}.''}
\url{https://theijf.org/donations-methodology}.

\leavevmode\vadjust pre{\hypertarget{ref-vikram2023large}{}}%
Vikram, Vasudev, Caroline Lemieux, and Rohan Padhye. 2023. {``{Can Large
Language Models Write Good Property-Based Tests?}''}
\url{https://arxiv.org/abs/2307.04346}.

\leavevmode\vadjust pre{\hypertarget{ref-white2023}{}}%
White, Jules, Quchen Fu, Sam Hays, Michael Sandborn, Carlos Olea, Henry
Gilbert, Ashraf Elnashar, Jesse Spencer-Smith, and Douglas C. Schmidt.
2023. {``{A Prompt Pattern Catalog to Enhance Prompt Engineering with
ChatGPT}.''} \url{https://arxiv.org/abs/2302.11382}.

\leavevmode\vadjust pre{\hypertarget{ref-wu2023}{}}%
Wu, Shijie, Ozan Irsoy, Steven Lu, Vadim Dabravolski, Mark Dredze,
Sebastian Gehrmann, Prabhanjan Kambadur, David Rosenberg, and Gideon
Mann. 2023. {``{BloombergGPT: A Large Language Model for Finance}.''}
\url{https://arxiv.org/abs/2303.17564}.

\leavevmode\vadjust pre{\hypertarget{ref-yong2023}{}}%
Yong, Gunwoo, Kahyun Jeon, Daeyoung Gil, and Ghang Lee. 2023. {``{Prompt
engineering for zero-shot and few-shot defect detection and
classification using a visual-language pretrained model}.''}
\emph{{Computer-Aided Civil and Infrastructure Engineering}} 38 (11):
1536--54. https://doi.org/\url{https://doi.org/10.1111/mice.12954}.

\leavevmode\vadjust pre{\hypertarget{ref-githubcopilot}{}}%
{``{Your AI pair programmer}.''} n.d.
\url{https://github.com/features/copilot}; {GitHub}.

\leavevmode\vadjust pre{\hypertarget{ref-yu2023}{}}%
Yu, Yue, Yuchen Zhuang, Jieyu Zhang, Yu Meng, Alexander Ratner, Ranjay
Krishna, Jiaming Shen, and Chao Zhang. 2023. {``{Large Language Model as
Attributed Training Data Generator: A Tale of Diversity and Bias}.''}
\url{https://arxiv.org/abs/2306.15895}.

\leavevmode\vadjust pre{\hypertarget{ref-zhang20}{}}%
Zhang, Lei, Sean Howard, Tom Montpool, Jessica Moore, Krittika Mahajan,
and Andriy Miranskyy. 2023. {``{Automated data validation: An industrial
experience report}.''} \emph{{The Journal of Systems \& Software}} 197:
111573. https://doi.org/\url{https://doi.org/10.1016/j.jss.2022.111573}.

\leavevmode\vadjust pre{\hypertarget{ref-zhao2021}{}}%
Zhao, Tony Z., Eric Wallace, Shi Feng, Dan Klein, and Sameer Singh.
2021. {``{Calibrate Before Use: Improving Few-Shot Performance of
Language Models}.''} \url{https://arxiv.org/abs/2102.09690}.

\end{CSLReferences}



\end{document}
